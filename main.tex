\documentclass[a4paper, 10pt]{article}

\usepackage{graphicx}
\usepackage{color}
\usepackage{tikz}
\usepackage{pgfplots}
\usepackage{pgf-umlsd}
\usepackage{ifthen}

\begin{document}

\begin{figure}
	\noindent\resizebox{\textwidth}{!}{
	\begin{tikzpicture}
		\draw[use as bounding box, transparent] (-1.8,-1.8) rectangle (17.2, 3.2);

		% Define the macro.
		% 1st argument: Height and width of the layer rectangle slice.
		% 2nd argument: Depth of the layer slice
		% 3rd argument: X Offset --> use it to offset layers from previously drawn layers.
		% 4th argument: Options for filldraw.
		% 5th argument: Text to be placed below this layer.
		% 6th argument: Y Offset --> Use it when an output needs to be fed to multiple layers that are on the same X offset.

		\newcommand{\networkLayer}[7]{
			\def\a{#1} % Used to distinguish input resolution for current layer.
			\def\b{0.02}
			\def\c{#2} % Width of the cube to distinguish number of input channels for current layer.
			\def\t{#3} % X offset for current layer.
			\def\d{#4} % Y offset for current layer.
			\def\z{#5} % Z offset for current layer.
			% Draw the layer body.
			\draw[line width=0.3mm](\c+\t,0+\z,\d) -- (\c+\t,\a+\z,\d) -- (\t,\a+\z,\d);                                                      % back plane
			\draw[line width=0.3mm](\t,0+\z,\a+\d) -- (\c+\t,0+\z,\a+\d) node[midway,below] {#7} -- (\c+\t,\a+\z,\a+\d) -- (\t,\a+\z,\a+\d) -- (\t,0+\z,\a+\d); % front plane
			\draw[line width=0.3mm](\c+\t,0+\z,\d) -- (\c+\t,0+\z,\a+\d);
			\draw[line width=0.3mm](\c+\t,\a+\z,\d) -- (\c+\t,\a+\z,\a+\d);
			\draw[line width=0.3mm](\t,\a+\z,\d) -- (\t,\a+\z,\a+\d);
			
			% Recolor visible surfaces
			\filldraw[#6] (\t+\b,\b+\z,\a+\d) -- (\c+\t-\b,\b+\z,\a+\d) -- (\c+\t-\b,\a-\b+\z,\a+\d) -- (\t+\b,\a-\b+\z,\a+\d) -- (\t+\b,\b+\z,\a+\d); % front plane
			\filldraw[#6] (\t+\b,\a+\z,\a-\b+\d) -- (\c+\t-\b,\a+\z,\a-\b+\d) -- (\c+\t-\b,\a+\z,\b+\d) -- (\t+\b,\a+\z,\b+\d);
			
			% Colored slice.
			\ifthenelse {\equal{#6} {}}
			{} % Do not draw colored slice if #4 is blank.
			{\filldraw[#6] (\c+\t,\b+\z,\a-\b+\d) -- (\c+\t,\b+\z,\b+\d) -- (\c+\t,\a-\b+\z,\b+\d) -- (\c+\t,\a-\b+\z,\a-\b+\d);} % Else, draw a colored slice.
		}

		% INPUT
		\node[] (input image) at (-3.75,0.5) {\includegraphics[height=30mm]{lenna.png}};
		\networkLayer{3.0}{0.03}{-0.5}{0.0}{0.0}{color=gray!80}{}

		% ENCODER
		\networkLayer{3.0}{0.1}{0.0}{0.0}{0.0}{color=white}{conv}    % S1
		\networkLayer{3.0}{0.1}{0.2}{0.0}{0.0}{color=white}{}        % S2
		\networkLayer{2.5}{0.2}{0.6}{0.0}{0.0}{color=white}{conv}    % S1
		\networkLayer{2.5}{0.2}{0.9}{0.0}{0.0}{color=white}{}        % S2
		\networkLayer{2.0}{0.4}{1.3}{0.0}{0.0}{color=white}{conv}    % S1
		\networkLayer{2.0}{0.4}{1.8}{0.0}{0.0}{color=white}{}        % S2
		\networkLayer{1.5}{0.8}{2.3}{0.0}{0.0}{color=white}{conv}    % S1
		\networkLayer{1.5}{0.8}{3.2}{0.0}{0.0}{color=white}{}        % S2
		\networkLayer{1.0}{1.5}{4.1}{0.0}{0.0}{color=white}{conv}    % S1
		\networkLayer{1.0}{1.5}{5.7}{0.0}{0.0}{color=white}{}        % S2

		% DECODER
		\networkLayer{1.0}{1.5}{7.7} {0.0}{0.0}{color=white}{deconv} % S1
		\networkLayer{1.0}{1.5}{9.3} {0.0}{0.0}{color=white}{}       % S2
		\networkLayer{1.5}{0.8}{11.2}{0.0}{0.0}{color=white}{deconv} % S1
		\networkLayer{1.5}{0.8}{12.1}{0.0}{0.0}{color=white}{}       % S2
		\networkLayer{2.0}{0.4}{13.3}{0.0}{0.0}{color=white}{}       % S1
		\networkLayer{2.0}{0.4}{13.8}{0.0}{0.0}{color=white}{}       % S2
		\networkLayer{2.5}{0.2}{14.8}{0.0}{0.0}{color=white}{}       % S1
		\networkLayer{2.5}{0.2}{15.1}{0.0}{0.0}{color=white}{}       % S2
		\networkLayer{3.0}{0.1}{15.8}{0.0}{0.0}{color=white}{}       % S1
		\networkLayer{3.0}{0.1}{16}  {0.0}{0.0}{color=white}{}       % S2

		% OUTPUT
		\networkLayer{3.0}{0.05}{17}{0.0}{0.0}{color=red!40}{}          % Pixelwise segmentation with classes.
		
			\networkLayer{3.0}{0.05}{17}{0.0}{7.0}{color=blue!40}{For Z}    
				\networkLayer{3.0}{0.05}{17}{0.0}{-7.0}{color=blue!40}{For Z}    
		\node[] (output image) at (18,0.5) {\includegraphics[height=30mm]{vermeer.jpg}};


	\end{tikzpicture}
	}
	\caption{Example CNN.}
	\label{fig:cnn}
\end{figure}

\end{document}
