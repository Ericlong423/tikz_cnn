\documentclass[a4paper, 10pt]{article}

\usepackage{graphicx}
\usepackage{color}
\usepackage{tikz}
\usepackage{pgfplots}
\usepackage{pgf-umlsd}
\usepackage{ifthen}
\usepackage[]{fp}

\FPset{totalOffset}{0}

\begin{document}

\begin{figure}
	\noindent\resizebox{\textwidth}{!}{
	\begin{tikzpicture}
		\draw[use as bounding box, transparent] (-1.8,-1.8) rectangle (17.2, 3.2);

		\newcommand{\networkLayer}[9]{
			% Define the macro.
			% 1st argument: Height and width of the layer rectangle slice.
			% 2nd argument: Depth of the layer slice
			% 3rd argument: X Offset --> use it to offset layers from previously drawn layers.
			% 4th argument: Y Offset --> Use it when an output needs to be fed to multiple layers that are on the same X offset.
			% 5th argument: Z Offset --> Use to offset layers from previous 
			% 6th argument: Options for filldraw.
			% 7th argument: Text to be placed below this layer.
			% 8th argument: Name of coordinates. When name = "start" this resets the offset counter
			% 9th argument: list of nodes to connect to (previous layers)
			\xdef\totalOffset{\totalOffset}
 			\ifthenelse{\equal{#8} {start}}
 			{\FPset{totalOffset}{0}}
 			{}
 			\FPeval\currentOffset{0+(totalOffset)+(#3)}

			\def\hw{#1} % Used to distinguish input resolution for current layer.
			\def\b{0.02}
			\def\c{#2} % Width of the cube to distinguish number of input channels for current layer.
			\def\x{\currentOffset} % X offset for current layer.
			\def\y{#4} % Y offset for current layer.
			\def\z{#5} % Z offset for current layer.
			\def\inText{#7}

            % Define references to points on the cube surfaces
            \coordinate (#8_front) at  (\x+\c  , \z      , \y);
            \coordinate (#8_back) at   (\x     , \z      , \y);
            \coordinate (#8_top) at    (\x+\c/2, \z+\hw/2, \y);
            \coordinate (#8_bottom) at (\x+\c/2, \z-\hw/2, \y);
            
 			% Define cube coords
			\coordinate (blr) at (\c+\x,  -\hw/2+\z,  -\hw/2+\y); %back lower right
			\coordinate (bur) at (\c+\x,   \hw/2+\z,  -\hw/2+\y); %back upper right
			\coordinate (bul) at (0 +\x,   \hw/2+\z,  -\hw/2+\y); %back upper left
			\coordinate (fll) at (0 +\x,  -\hw/2+\z,   \hw/2+\y); %front lower left
			\coordinate (flr) at (\c+\x,  -\hw/2+\z,   \hw/2+\y); %front lower right
			\coordinate (fur) at (\c+\x,   \hw/2+\z,   \hw/2+\y); %front upper right
			\coordinate (ful) at (0 +\x,   \hw/2+\z,   \hw/2+\y); %front upper left

            % Draw connections from other points to the back of this node
            \ifthenelse{\equal{#9} {}}
 			{}{
 			    \foreach \val in #9
 			        \draw[line width=0.3mm] (\val)--(#8_back);
 			}
 			
			% Draw the layer body.
			% back plane
			\draw[line width=0.3mm](blr) -- (bur) -- (bul);
			% front plane
			\draw[line width=0.3mm](fll) -- (flr) node[midway,below] {\inText} -- (fur) -- (ful) -- (fll);
			\draw[line width=0.3mm](blr) -- (flr);
			\draw[line width=0.3mm](bur) -- (fur);
			\draw[line width=0.3mm](bul) -- (ful);

			% Recolor visible surfaces
			% front plane
			\filldraw[#6] ($(fll)+(\b,\b,0)$) -- ($(flr)+(-\b,\b,0)$) -- ($(fur)+(-\b,-\b,0)$) -- ($(ful)+(\b,-\b,0)$) -- ($(fll)+(\b,\b,0)$);
			\filldraw[#6] ($(ful)+(\b,0,-\b)$) -- ($(fur)+(-\b,0,-\b)$) -- ($(bur)+(-\b,0,\b)$) -- ($(bul)+(\b,0,\b)$);

			% Colored slice.
			\ifthenelse {\equal{#6} {}}{} % Do not draw colored slice if #4 is blank.
			% Else, draw a colored slice.
			{\filldraw[#6] ($(flr)+(0,\b,-\b)$) -- ($(blr)+(0,\b,\b)$) -- ($(bur)+(0,-\b,\b)$) -- ($(fur)+(0,-\b,-\b)$);}

			\FPeval\totalOffset{0+(currentOffset)+\c}
		}

		% INPUT
		\networkLayer{3.0}{0.03}{0.0}{0.0}{0.0}{color=gray!80}{}{start}

		% ENCODER
		\networkLayer{3.0}{0.1}{0.5}{0.0}{0.0}{color=white}{conv}{}    % S1
		\networkLayer{3.0}{0.1}{0.1}{0.0}{0.0}{color=white}{}{}        % S2
		\networkLayer{2.5}{0.2}{0.1}{0.0}{0.0}{color=white}{conv}{}    % S1
		\networkLayer{2.5}{0.2}{0.1}{0.0}{0.0}{color=white}{}{}        % S2
		\networkLayer{2.0}{0.4}{0.1}{0.0}{0.0}{color=white}{conv}{}    % S1
		\networkLayer{2.0}{0.4}{0.1}{0.0}{0.0}{color=white}{}{}        % S2
		\networkLayer{1.5}{0.8}{0.1}{0.0}{0.0}{color=white}{conv}{}    % S1
		\networkLayer{1.5}{0.8}{0.1}{0.0}{0.0}{color=white}{}{}        % S2
		\networkLayer{1.0}{1.5}{0.1}{0.0}{0.0}{color=white}{conv}{}    % S1
		\networkLayer{1.0}{1.5}{0.1}{0.0}{0.0}{color=white}{}{}        % S2

		% DECODER
		\networkLayer{1.0}{1.5}{0.3}{0.0}{0.0}{color=white}{deconv}{} % S1
		\networkLayer{1.0}{1.5}{0.1}{0.0}{0.0}{color=white}{}{}       % S2
		\networkLayer{1.5}{0.8}{0.1}{0.0}{0.0}{color=white}{deconv}{} % S1
		\networkLayer{1.5}{0.8}{0.1}{0.0}{0.0}{color=white}{}{}       % S2
		\networkLayer{2.0}{0.4}{0.1}{0.0}{0.0}{color=white}{}{}       % S1
		\networkLayer{2.0}{0.4}{0.1}{0.0}{0.0}{color=white}{}{}       % S2
		\networkLayer{2.5}{0.2}{0.1}{0.0}{0.0}{color=white}{}{}       % S1
		\networkLayer{2.5}{0.2}{0.1}{0.0}{0.0}{color=white}{}{}       % S2
		\networkLayer{3.0}{0.1}{0.1}{0.0}{0.0}{color=white}{}{}       % S1
		\networkLayer{3.0}{0.1}{0.1}{0.0}{0.0}{color=white}{}{}       % S2

		% OUTPUT
		\networkLayer{3.0}{0.05}{0.9}{0.0}{0.0}{color=red!40}{}{}          % Pixelwise segmentation with classes.
		\networkLayer{3.0}{0.05}{-0.05}{0.0}{7.0}{color=blue!40}{For Z}{}


	\end{tikzpicture}
	}
	\caption{Example CNN.}
	\label{fig:cnn}
\end{figure}

\end{document}
